%%%%%%%%%%%%%%%%%%%%%%%%%%%%%%%%%%%%%%%%%%%%%%%%%%%%%%%%%%%%%%%%%%%%%%%%%%%%%%%
%%
%% FACHHOCHSCHULE SALZBURG GMBH
%% Informationstechnik und System-Management
%%
%% Salzburg University of Applied Sciences
%% Information Technologies and Systems Management
%%
%%%%%%%%%%%%%%%%%%%%%%%%%%%%%%%%%%%%%%%%%%%%%%%%%%%%%%%%%%%%%%%%%%%%%%%%%%%%%%%
%%
%% Bachelor Thesis 1
%% Aufbau eines Mobile IPv6 Szenarios im Netzwerklabor
%% LaTeX template
%%
%% Zusammenfassung und Ausblick
%%
%% Riccardo Martin, Michael Pfn�r, Daniel Zotter
%%%%%%%%%%%%%%%%%%%%%%%%%%%%%%%%%%%%%%%%%%%%%%%%%%%%%%%%%%%%%%%%%%%%%%%%%%%%%%%

\chapter{Zusammenfassung und Ausblick}
%%%%%%%%%%%%%%%%%%%%%%%%%%%%%%%%%%%%%%
Die "`Conclusio"' dient der Abrundung der wissenschaftlichen Bachelorarbeit. Sie umfasst in komprimierter Form die wesentlichen Aussagen zur L�sung der Aufgabe bzw. die knappe Darstellung von erarbeiteten Thesen. Der/die VerfasserIn kann hier deutlich machen, dass das in der Einleitung angek�ndigte Anliegen der Arbeit erreicht worden ist. 

Weiters gibt das abschlie�ende Kapitel Raum f�r kritische Anmerkungen und kann dar�ber hinaus dazu genutzt werden, den LeserInnen Informationen �ber zu erwartende Entwicklungen auf dem behandelten Themengebiet zu liefern.











