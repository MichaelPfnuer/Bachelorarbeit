%%%%%%%%%%%%%%%%%%%%%%%%%%%%%%%%%%%%%%%%%%%%%%%%%%%%%%%%%%%%%%%%%%%%%%%%%%%%%%%
%%
%% FACHHOCHSCHULE SALZBURG GMBH
%% Informationstechnik und System-Management
%%
%% Salzburg University of Applied Sciences
%% Information Technologies and Systems Management
%%
%%%%%%%%%%%%%%%%%%%%%%%%%%%%%%%%%%%%%%%%%%%%%%%%%%%%%%%%%%%%%%%%%%%%%%%%%%%%%%%
%%
%% Bachelor Thesis 1
%% Aufbau eines Mobile IPv6 Szenarios im Netzwerklabor
%% LaTeX template
%%
%% Zusammenfassung und Ausblick
%%
%% Riccardo Martin, Michael Pfn�r, Daniel Zotter
%%%%%%%%%%%%%%%%%%%%%%%%%%%%%%%%%%%%%%%%%%%%%%%%%%%%%%%%%%%%%%%%%%%%%%%%%%%%%%%

\chapter{Zusammenfassung und Ausblick}
\label{Zusammenfassung}
Diese Bachelorarbeit befasste sich mit dem Aufbau eines Mobile IPv6 Szenarios im Netzwerklabor. Mobile IPv6 ist ein Protokoll, welches es erm�glicht eine feste IPv6 Adresse einem mobilen Endger�t zuzuweisen und diese auch bei einem Netzwechsel zu behalten.
 
Um die Vorz�ge von Mobile IPv6 zu erl�utern wurde eine genaue Beschreibung von Mobile IPv6 aufgef�hrt. Au�erdem wurde ein Vergleich zwischen Mobile IPv4 und Mobile IPv6 gezeigt um die �nderungen darzustellen.  
Ein gro�er Vorteil von IPv6 ist, dass die M�glichkeiten zur Adressierung von Systemen von IPv6 gegen�ber IPv4 stark gestiegen sind. Somit kann man dem prognostizierten Verbrauch von IP Adressen entgegenwirken. Jedoch musste festgestellt werden, dass bei derzeitigen Entwicklungsstand von Anwendungen und Hardware eine Einf�hrung von Mobile IPv6 aufgrund fehlender Hardware- und Softwareunterst�tzungen nicht m�glich ist.

Nach einer allgemeineren Untersuchung bez�glich der Systemunterst�tzung von IPv6, muss festgehalten werden, dass auch hier Diskrepanzen bestehen. In den letzten Jahren hat sich viel getan in Richtung IPv6, dennoch gibt es weiter viele L�cken in den Betriebssystemen und in den Routern, welche die Implementierung von IPv6 und insbesondere von Mobile IPv6 nahezu unm�glich machen. So stellte sich zum Beispiel bei dem Versuch einem WLAN Interface eine IPv6 Adresse zuzuweisen heraus, dass dieses IP Adressen der Version 6 nicht unterst�tzt. Jedoch kann davon ausgegangen werden, dass dieses Problem in naher Zukunft durch Aktualisierungen der Hardware von den Herstellern behoben sein wird. Fr�her oder sp�ter werden zuk�nftige Konfigurationen von IP Adressen �berwiegen auf IPv6 Adressen errichtet.

Im Netzwerklabor wurden nach auftretenden Problemen drei Testszenarien aufgebaut. Jedes der Szenarien konnte verschiedene Probleme beheben. Am Ende jedoch muss festgehalten werden, dass die fehlende Implementierung von Mobile IPv6 eine Umsetzung unm�glich macht. Anhand der dabei gewonnen Erkenntnisse und Erfahrungen l�sst sich sagen, dass zum Beispiel die M�glichkeit der Adressvergabe sowie das Subnetting einen gro�en Vorteil bietet.

Ein Ausblick auf zuk�nftige Implementierungen bezieht sich haupts�chlich auf die Anschaffung von aktuelleren Hardwareversionen, welche IPv6 im allgemeinen und Mobile IPv6 besser unterst�tzen und die Implementierung von Mobile IPv6 unter Linux. Linux bietet unter anderem Kernelversionen, welche es erlauben Binding Updates zu erstellen und zu senden. Die Umsetzung mit Linux wurde in dieser Arbeit jedoch nicht ber�cksichtigt, k�nnte aber eine L�sung bieten.

Abschlie�end l�sst sich sagen, dass trotz der zur Zeit mangelhaften Systemunterst�tzungen die Technologie von Mobile IPv6 ein gro�es Potential besitzt. Die unzureichende Implementierung von Mobile IPv6 l�sst sich wahrscheinlich auf die zu heutigen Zeitpunkt fehlende Notwendigkeit zur�ckf�hren. Auf lange Sicht wird die Umstellung zu IPv6 und die damit verbundenen Dienste immer weiter voranschreiten. Es wird also irgendwann zu dem Zeitpunkt kommen, an dem diese Technologien weiter entwickelt und auch umgesetzt werden.              










