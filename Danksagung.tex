%%%%%%%%%%%%%%%%%%%%%%%%%%%%%%%%%%%%%%%%%%%%%%%%%%%%%%%%%%%%%%%%%%%%%%%%%%%%%%%
%%
%% FACHHOCHSCHULE SALZBURG GMBH
%% Informationstechnik und System-Management
%%
%% Salzburg University of Applied Sciences
%% Information Technologies and Systems Management
%%
%%%%%%%%%%%%%%%%%%%%%%%%%%%%%%%%%%%%%%%%%%%%%%%%%%%%%%%%%%%%%%%%%%%%%%%%%%%%%%%
%%
%% Bachelor Thesis 1
%% Aufbau eines Mobile IPv6 Szenarios im Netzwerklabor
%% LaTeX template
%%
%% Danksagung
%%
%% Riccardo Martin, Michael Pfn�r, Daniel Zotter
%%%%%%%%%%%%%%%%%%%%%%%%%%%%%%%%%%%%%%%%%%%%%%%%%%%%%%%%%%%%%%%%%%%%%%%%%%%%%%%



\section*{Danksagung}

Zun�chst m�chten wir uns an dieser Stelle bei all denjenigen bedanken, die uns w�hrend der Anfertigung dieser Bachelorarbeit unterst�tzt haben.\\

Ganz besonders danken m�chten wir in erster Linie unserer Betreuerin, Frau FH-Ass.~Prof.~Dipl.~Phys.~Judith Schwarzer, f�r ihre ausgiebige Unterst�tzung. Durch stetiges Hinterfragen und konstruktive Kritik verhalf sie uns zu einer durchdachten Herangehensweise und Umsetzung. Dank ihrer Erfahrung im Bereich der Netzwerktechnik konnte sie uns immer wieder in unserer Recherche und bei unseren Fragen unterst�tzen. Vielen Dank f�r Zeit und M�hen, die Sie in unsere Arbeit investiert haben.\\

Auch m�chten wir uns bei der Fachhochschule Salzburg bedanken, die das ben�tigte Equipment und die R�umlichkeiten zur Verf�gung gestellt hat.\\ 

\newpage