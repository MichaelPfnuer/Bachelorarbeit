%%%%%%%%%%%%%%%%%%%%%%%%%%%%%%%%%%%%%%%%%%%%%%%%%%%%%%%%%%%%%%%%%%%%%%%%%%%%%%%
%%
%% FACHHOCHSCHULE SALZBURG GMBH
%% Informationstechnik und System-Management
%%
%% Salzburg University of Applied Sciences
%% Information Technologies and Systems Management
%%
%%%%%%%%%%%%%%%%%%%%%%%%%%%%%%%%%%%%%%%%%%%%%%%%%%%%%%%%%%%%%%%%%%%%%%%%%%%%%%%
%%
%% Bachelor Thesis 1
%% Aufbau eines Mobile IPv6 Szenarios im Netzwerklabor
%% LaTeX template
%%
%% Praktischer Teil
%%
%% Riccardo Martin, Michael Pfn�r, Daniel Zotter
%%%%%%%%%%%%%%%%%%%%%%%%%%%%%%%%%%%%%%%%%%%%%%%%%%%%%%%%%%%%%%%%%%%%%%%%%%%%%%%


\chapter{Praktischer Teil}

Die Darstellung der Untersuchungs-, Anwendungs- oder Umsetzungs-Methoden und die Beschreibung der Umsetzung sollen die Wege aufzeigen, wie man zu bestimmten Ergebnissen gelangt ist. Es ist nachzuweisen, dass die dargestellten Implementierungen, Analysen und abgeleiteten Schlussfolgerungen nicht nur Frucht eigener kreativer �berlegungen sind, sondern dass sie auf einer soliden Informationsbasis und einem nachvollziehbaren Analyseverfahren beruhen.

\section{Verwendete Materialien}



\section{Physischer Aufbau}


