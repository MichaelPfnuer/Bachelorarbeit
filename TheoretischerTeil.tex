%%%%%%%%%%%%%%%%%%%%%%%%%%%%%%%%%%%%%%%%%%%%%%%%%%%%%%%%%%%%%%%%%%%%%%%%%%%%%%%
%%
%% FACHHOCHSCHULE SALZBURG GMBH
%% Informationstechnik und System-Management
%%
%% Salzburg University of Applied Sciences
%% Information Technologies and Systems Management
%%
%%%%%%%%%%%%%%%%%%%%%%%%%%%%%%%%%%%%%%%%%%%%%%%%%%%%%%%%%%%%%%%%%%%%%%%%%%%%%%%
%%
%% Bachelor Thesis 1
%% Aufbau eines Mobile IPv6 Szenarios im Netzwerklabor
%% LaTeX template
%%
%% Theoretischer Teil
%%
%% Riccardo Martin, Michael Pfn�r, Daniel Zotter
%%%%%%%%%%%%%%%%%%%%%%%%%%%%%%%%%%%%%%%%%%%%%%%%%%%%%%%%%%%%%%%%%%%%%%%%%%%%%%%


\chapter{Theoretischer Teil}
\label{chapter_grundlagen}

%Der Kernteil der Arbeit beginnt mit einer Darstellung der Grundlagen. Dieser, in der Regel rein theoretische Abschnitt, beinhaltet Begriffsbestimmungen, Beschreibungen zur Methodik und zum verwendeten Material, zu Hard- und Software sowie Begriffsabgrenzungen und anderweitigen Grundlagen ("`Material und Methode"'), welche zum Verst�ndnis der nachfolgenden Ausf�hrungen notwendig sind.
Mobile IPv6 ist ein Protokoll, dass von der IETF entwickelt wurden welches es erm�glicht, eine feste IPv6 Adresse einem mobilen Endger�t zuzuweisen und diese auch bei Netzwechseln zu behalten.  
Im folgenden Kapitel wird auf die theoretische Funktionsweise von \textbf{Mobile IPv6} eingegangen, sowie den Unterschied zwischen den Versionen IPv4 und IPv6. In diesem Teil werden einige Fachbegriffe in Bezug auf Mobile IPv6 verwendet, welche f�r das Verst�ndnis der Funktionsweise wichtig sind. Diese Begriffe werden in Abschnitt \ref{section_Begriffserklaerung_MobileIPV6} kurz erkl�rt\cite{RFC_Mobile_IPv6}. 
\section{Begriffsdefinition}
\label{section_Begriffserklaerung_MobileIPV6}

\subsection*{Home Adresse}
Die \textit{Home Adresse} ist eine Unicast Adresse welche dem Mobilen Knoten zugewiesen wird, sie wird als permanente Adresse dieses Knoten benutzt. Diese befindet sich innerhalb des Home Links des mobilen Knoten. IP Routing Mechanismen schicken an die Home Adresse gerichtete Pakete an den Home Link. Falls es mehrere Pr�fixe auf dem Home Link gibt, kann ein Mobiler Knoten auch mehrere Home Adressen besitzen.

\subsection*{Home Subnetz Pr�fix}
Unter \textit{Home Subnetz Pr�fix} versteht man das IP-Subnetzpr�fix, dass der Home Adresse des mobilen Knoten entspricht.

\subsection*{Home Link}
Der \textit{Home Link}, ist der Link an welchem das Home-Subnetzpr�fix definiert ist.

\subsection*{Mobiler node}
Ein \textit{Mobiler node} ist ein Knoten, welcher seinen Standort wechseln kann (z.B. Laptop, Mobil Telefon etc.). Dieser Knoten bleibt aber auch unter seiner Home Adresse erreichbar, wenn er von seinem \textit{Heimnetz A} in ein \textit{Fremdnetz B} wechselt.

\subsection*{Correspondent node}
Ein \textit{Correspondent node} ist ein peer (gleichberechtigter Teilnehmer) Knoten mit dem der mobile Knoten kommuniziert. Der correspondant node kann ein mobiler oder station�rer Knoten sein.

\subsection*{Foreign Subnet Pr�fix}
Unter \textit{Foreign Subnet Pr�fix} versteht man jedes Subnet Pr�fix, das nicht dem Home Subnet Pr�fix des mobilen Knotens entspricht.

\subsection*{Foreign Link}
Ist jeder Link, der nicht dem Home Link des mobilen Knotens entspricht.

\subsection*{Care-of Adresse}
Die \textit{Care-of Adresse} ist eine Unicast Adresse, die dem mobilen Knoten in einem fremden Netz zugewiesen wird. Ein mobiler Knoten kann auch mehrere Care-of Adressen besitzen (z.B. mit verschiedenen Pr�fixes), die Care-of Adresse mit der er bei seinem \textit{Home Agent} registriert ist, wird als \textit{\glqq Primary\grqq Care-of Adresse} bezeichnet.

\subsection*{Home Agent}
Als \textit{Home Agent} wird der Router bezeichnet der sich am \textit{Home Link} des mobilen Knotens befindet und wo die aktuelle \textit{Care-of Adresse} des mobilen Knoten registriert ist. Wenn sich der mobile Knoten nicht im Heimnetz befindet, f�ngt der \textit{Home Agent} die Pakete, die an die Home Adresse des mobilen Knoten im Heimnetz gerichtet sind ab, \glqq verpackt\grqq diese und sendet sie �ber einen Tunnel an die registrierte \textit{Care-of Adresse} des mobilen Knoten.

\subsection*{Binding}
Als \textit{Binding} versteht man die Zuordnung der \textit{Home Adresse} des mobilen Knotens, der \textit{Care-of Adresse} des mobilen Knotens f�r die noch verbleibende lifetime.

\subsection*{Registrierung}
Unter \textit{Registrierung} versteht man, wenn ein Binding Update von einem mobilen Knoten an seinen Home Agent oder an einen Corresponding Node  geschickt und von diesen registriert wird.

\subsection*{Binding Authentisierung}
Damit ein Corresponding Node weiss, dass ein Absender berechtigt ist das Binding zu �ndern, muss eine Registrierung bei einem Corresponding Node autorisiert werden.\newpage

%\section{Mobile IPv6}
\label{section_TheorieMobileIPv6}

%\subsection*{Beispiel}



%\section{Vergleich Mobile IPv4 zu Mobile IPv6}









