%%%%%%%%%%%%%%%%%%%%%%%%%%%%%%%%%%%%%%%%%%%%%%%%%%%%%%%%%%%%%%%%%%%%%%%%%%%%%%%
%%
%% FACHHOCHSCHULE SALZBURG GMBH
%% Informationstechnik und System-Management
%%
%% Salzburg University of Applied Sciences
%% Information Technologies and Systems Management
%%
%%%%%%%%%%%%%%%%%%%%%%%%%%%%%%%%%%%%%%%%%%%%%%%%%%%%%%%%%%%%%%%%%%%%%%%%%%%%%%%
%%
%% Bachelor Thesis 1
%% Aufbau eines Mobile IPv6 Szenarios im Netzwerklabor
%% LaTeX template
%%
%% Theoretischer Teil
%%
%% Riccardo Martin, Michael Pfn�r, Daniel Zotter
%%%%%%%%%%%%%%%%%%%%%%%%%%%%%%%%%%%%%%%%%%%%%%%%%%%%%%%%%%%%%%%%%%%%%%%%%%%%%%%


\chapter{Theoretischer Teil}
\label{chapter_grundlagen}

Der Kernteil der Arbeit beginnt mit einer Darstellung der Grundlagen. Dieser, in der Regel rein theoretische Abschnitt, beinhaltet Begriffsbestimmungen, Beschreibungen zur Methodik und zum verwendeten Material, zu Hard- und Software sowie Begriffsabgrenzungen und anderweitigen Grundlagen ("`Material und Methode"'), welche zum Verst�ndnis der nachfolgenden Ausf�hrungen notwendig sind.
h�������
\section{Mobile IPv6}


\subsection*{Beispiel}



\section{Vergleich Mobile IPv4 zu Mobile IPv6}









