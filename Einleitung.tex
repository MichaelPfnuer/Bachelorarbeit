%%%%%%%%%%%%%%%%%%%%%%%%%%%%%%%%%%%%%%%%%%%%%%%%%%%%%%%%%%%%%%%%%%%%%%%%%%%%%%%
%%
%% FACHHOCHSCHULE SALZBURG GMBH
%% Informationstechnik und System-Management
%%
%% Salzburg University of Applied Sciences
%% Information Technologies and Systems Management
%%
%%%%%%%%%%%%%%%%%%%%%%%%%%%%%%%%%%%%%%%%%%%%%%%%%%%%%%%%%%%%%%%%%%%%%%%%%%%%%%%
%%
%% Bachelor Thesis 1
%% Aufbau eines Mobile IPv6 Szenarios im Netzwerklabor
%% LaTeX template
%%
%% Einleitung
%%
%% Riccardo Martin, Michael Pfn�r, Daniel Zotter
%%%%%%%%%%%%%%%%%%%%%%%%%%%%%%%%%%%%%%%%%%%%%%%%%%%%%%%%%%%%%%%%%%%%%%%%%%%%%%%


\chapter{Einleitung}
\label{chapter_einfuehrung}
%%%%%%%%%%%%%%%%%%%%%%%%%%%%%%%%
Die Einleitung stellt das Thema vor, begr�ndet die Themenwahl-/stellung, indem sie Relevanz, Motivation und Zielsetzung der Arbeit erl�utert. Um den LeserInnen einen �berblick �ber die Arbeit zu verschaffen und die eigene inhaltliche Vorgangsweise transparent zu machen, ist der Aufbau der gesamten Arbeit kurz zu schildern.




\section{Motivation und Aufgabenstellung}
\label{section_motivation}

Diese Bachelorarbeit wurde im Rahmen einer Kooperation ...
wie in \ref{section_motivation} 

\section{Aufbau und Kapitel�bersicht}
\label{section_kapitel}

In Kapitel \ref{chapter_grundlagen} werden die theoretischen Grundlagen ...
\author{}




