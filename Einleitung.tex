%%%%%%%%%%%%%%%%%%%%%%%%%%%%%%%%%%%%%%%%%%%%%%%%%%%%%%%%%%%%%%%%%%%%%%%%%%%%%%%
%%
%% FACHHOCHSCHULE SALZBURG GMBH
%% Informationstechnik und System-Management
%%
%% Salzburg University of Applied Sciences
%% Information Technologies and Systems Management
%%
%%%%%%%%%%%%%%%%%%%%%%%%%%%%%%%%%%%%%%%%%%%%%%%%%%%%%%%%%%%%%%%%%%%%%%%%%%%%%%%
%%
%% Bachelor Thesis 1
%% Aufbau eines Mobile IPv6 Szenarios im Netzwerklabor
%% LaTeX template
%%
%% Einleitung
%%
%% Riccardo Martin, Michael Pfn�r, Daniel Zotter
%%%%%%%%%%%%%%%%%%%%%%%%%%%%%%%%%%%%%%%%%%%%%%%%%%%%%%%%%%%%%%%%%%%%%%%%%%%%%%%


\chapter{Einleitung}
\label{chapter_einfuehrung}
%%%%%%%%%%%%%%%%%%%%%%%%%%%%%%%%
%Die Einleitung stellt das Thema vor, begr�ndet die Themenwahl-/stellung, indem sie Relevanz, Motivation und Zielsetzung der Arbeit erl�utert. Um den LeserInnen einen �berblick �ber die Arbeit zu verschaffen und die eigene inhaltliche Vorgangsweise transparent zu machen, ist der Aufbau der gesamten Arbeit kurz zu schildern.
Mit der Einf�hrung des Internet Protkolls IPv6 im Jahre 1998 wurde ein Nachfolger f�r das bis zu diesem Zeitpunkt alleinig verwendete IPv4 auf den Weg gebracht. IPv6 soll als Nachfolger von IPv4 dieses in absehbarer Zeit abl�sen, was eine alleinige Nutzung der Version 6 des Internet Protokolls zur Folge hat.\\
Aus diesem Grund wird in der nachfolgenden Arbeit ein Einsatzbereich dieses Protokolles betrachtet.



\section{Motivation und Aufgabenstellung}
\label{section_motivation}

Das Thema \textbf{Aufbau eines Mobile IPv6 Szenarios im Netzwerklabor} wurde f�r diese Bachelor Arbeit gew�hlt, da die Anzahl mobiler Endger�te Ende 2014 schon \textit{7.9 Milliarden} betrug und in den n�chsten Jahren stetig steigen wird. Dies ist ein gewichtiger Grund warum die Anwendung von Mobile IPv6 und den daraus resultierenden Vorteilen in der Zukunft zunehmend Beachtung geschenkt werdern sollte. F�hrt man sich nur einmal vor Augen wie oft ein Mobilger�t einen Netzwechsel bei einer Fahrt mit dem Zug von M�nchen nach Hamburg vollzieht, so ist leicht ersichtlich, dass diese Technologie in Zukunft von enormer Bedeutung sein wird (genaue Erkl�rung der Funktionsweise in Abschnitt \ref{section_TheorieMobileIPv6}). Unter diese verschiedenen Gesichtspunkte, war es uns ein Anliegen, dieses Thema zu erarbeiten und zu vertiefen. 

\section{Aufbau und Kapitel�bersicht}
\label{section_kapitel}

Der Aufbau dieser Arbeit wie folgend gegliedert. In Kapitel 1 wird mit einer kurzen Einleitung auf das Theme hingef�hrt, sowie die Motivation f�r die Bearbeitung dieser Aufgabenstellung und die Aufgabenstellung selbst dargestellt. \\
Kapitel 2 befasst sich mit der theoretischen Erkl�rung von \textbf{Mobile IPv6} und der f�r das Verst�ndnis n�tigen Beschreibung einiger Fachbegriffe dieses Themas. Weiterhin wird ein kurzer Vergleich zwischen \textbf{Mobile IPv4} und \textbf{Mobiel IPv6} gezogen und die sich daraus ergebenden Vor- und Nachteile dargestellt.\\
In Kapitel 3 wird der physische Aufbau des Netzwerks, die dort verwendeten Materialien (Router, Switch etc.), eine Analyse der Hardware und der Software sowie die Implementation der Konfigurationen n�her beschrieben und dargestellt.\\
Im letzten Kapitel werden die Ergebnisse, welche sich ergaben noch einmal zusammengefasst und ein Ausblick in die weiter Zukunft beschrieben.\\
In Anhang sind zuletzt noch das Literaturverzeichnis, Abk�rzungsverzeichnis, Abbildungsverzeichnis, Tabellenverzeichnis und Quellcodeverzeichnis zu finden.




