%%%%%%%%%%%%%%%%%%%%%%%%%%%%%%%%%%%%%%%%%%%%%%%%%%%%%%%%%%%%%%%%%%%%%%%%%%%%%%%
%%
%% FACHHOCHSCHULE SALZBURG GMBH
%% Informationstechnik und System-Management
%%
%% Salzburg University of Applied Sciences
%% Information Technologies and Systems Management
%%
%%%%%%%%%%%%%%%%%%%%%%%%%%%%%%%%%%%%%%%%%%%%%%%%%%%%%%%%%%%%%%%%%%%%%%%%%%%%%%%
%%
%% Bachelor Thesis 1
%% Aufbau eines Mobile IPv6 Szenarios im Netzwerklabor
%% LaTeX template
%%
%% Kurzzusammenfassung und Abstract
%%
%% Riccardo Martin, Michael Pfn�r, Daniel Zotter
%%%%%%%%%%%%%%%%%%%%%%%%%%%%%%%%%%%%%%%%%%%%%%%%%%%%%%%%%%%%%%%%%%%%%%%%%%%%%%%


\section*{Kurzzusammenfassung}
\normalsize
\vspace{1em}

Diese Arbeit besch�ftigt sich mit Mobile IPv6 und dem Aufbau eines Prototypen. Sie gliedert sich in einen theoretischen und einen praktischen Teil.

Die Theorie besch�ftigt sich mit dem n�tigen Basiswissen �ber Mobile IPv6. Hier werden die verschiedenen Begriffsdefinitionen in der Arbeit aufgelistet und erkl�rt. Es wird auf die speziellen Mobile IPv6 Header die das Protokoll einf�hrt eingegangen und die generelle Funktionsweise von Mobile IPv6 beschrieben. Da der Umstieg von IPv4 auf IPv6 nur sehr langsam voran geht wurde auch ein Vergleich zwischen Mobile IPv4 und IPv6 in dieser Arbeit behandelt.
Au�erdem wurde noch eine Erweiterung des Mobile IPv6 Protokolls Namens Network Mobility NEMO betrachtet. Diese Erweiterung hatte sich im Praktischen Teil als M�glichkeit zur Realisierung des Prototypen angeboten.

Der Praktische Teil ist in die Beschreibung von 3 Versuchen aufgegliedert und einer folgenden Analyse der verwendeten Hard und Software sowie der Implementierung der einzelnen Versuche. Die Beschreibung der Versuche gliedert sich in deren Aufbau einer Grafik zum veranschaulichen und dem Ergebnis dieses Versuches. Der Aufbau des folgende Versuch war immer eine weiter Entwicklung des vorherigen. Die Analyse beleuchtet genauer die verwendete Hardware und besch�ftigt sich auch mit der Auswahl des richtigen IOS f�r die Umsetzung. Die Implementierung der einzelnen Versuche analysiert detaillierter Aufbau und Ergebnisse aus den einzelnen Versuchen. Zum Schluss gibt es noch eine Zusammenfassung sowie einen Ausblick in welcher die Erkenntnisse der Arbeit nochmal aufbereitet werden.           


\section*{Abstract}
\normalsize
\vspace{1em}


This is an example for a \emph{short} abstract.